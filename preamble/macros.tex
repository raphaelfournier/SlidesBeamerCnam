% <<< From Dr Bunsen
% adds reference to bottom right of corner of a slide
\usepackage[absolute,overlay]{textpos} % text references in slide corners
\newcommand\textref[1]{%
  \begin{textblock*}{\paperwidth}(0pt,0.99\textheight)
  \raggedleft \tiny{\emph{#1}}\hspace{.5em}
  \end{textblock*}}

% for drawing circles around numbers
% ex. \circled{1} Add some text here.
\usepackage{tikz}
\newcommand*\circled[1]{\tikz[baseline=(char.base)]{
            \node[shape=circle,draw,inner sep=2pt] (char) {#1};}}

% >>>

% <<< From Execushares
% disable "Figure:" in the captions
\setbeamertemplate{caption}{\tiny\insertcaption}
\setbeamertemplate{caption label separator}{}

\setbeamertemplate{section in toc}{%
\inserttocsectionnumber~|~\inserttocsection \par}

%\setbeamertemplate{subsection in toc}{
%-- \inserttocsubsection \par}
% >>>

% <<< custom variables
\newcommand\insertpromo{}
\newcommand\promo[1]{\renewcommand\insertpromo{#1}}
\newcommand\insertmatiere{}
\newcommand\matiere[1]{\renewcommand\insertmatiere{#1}}
\newcommand\insertannee{}
\newcommand\annee[1]{\renewcommand\insertannee{#1}}
\newcommand\insertemail{}
\newcommand\email[1]{\renewcommand\insertemail{#1}}
% >>>

% <<< materiel de cours
\newcommand{\caveat}[1]{
  \begin{itemize}
    \item[\color{darkred}$\spadesuit$] \alert{#1}
  \end{itemize}}

%% Import a figure in a slide
\newcommand{\figSlideWithSize}[2]{%
\begin{center}
   \includegraphics[width=#2]{#1}
\end{center}
}
\newcommand{\guill}[1]{<<\,#1\,>>}
\newcommand{\variable}[1]{{\bf \color{darkblue} #1}}
\definecolor{darkblue}{rgb}{.0,.0,.6}

\newcommand{\figSlide}[1]{%
\begin{center}
   \includegraphics{#1}
\end{center}
}

% for tables
\newcommand{\mcrot}[1]{\multicolumn{1}{l}{\rlap{\rotatebox{45}{#1}~}}} 
\newcommand*{\twoelementtable}[3][l]%
{%  
  \renewcommand{\arraystretch}{0.8}%
  \begin{tabular}[t]{@{}#1@{}}%
    #2\tabularnewline
    #3%
  \end{tabular}%
}

\newcommand{\yes}{~x~}
\newcommand{\nnn}{~-~}
\newcommand{\yTwo}{\multicolumn{2}{c}{~x~}}
\newcommand{\nTwo}{\multicolumn{2}{c}{~-~}}

% >>>

% <<< biblatex
\DefineBibliographyStrings{english}{
    and = {and}
  }
% >>>
% vim: set fdm=marker fmr=<<<,>>> fdl=0:fdc=2
